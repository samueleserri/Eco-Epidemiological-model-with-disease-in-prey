% $Header$

\documentclass{beamer}

% This file is a solution template for:

% - Talk at a conference/colloquium.
% - Talk length is about 20min.
% - Style is ornate.



% Copyright 2004 by Till Tantau <tantau@users.sourceforge.net>.
%
% In principle, this file can be redistributed and/or modified under
% the terms of the GNU Public License, version 2.
%
% However, this file is supposed to be a template to be modified
% for your own needs. For this reason, if you use this file as a
% template and not specifically distribute it as part of a another
% package/program, I grant the extra permission to freely copy and
% modify this file as you see fit and even to delete this copyright
% notice. 


\mode<presentation>
{
\usetheme{Frankfurt}
\usecolortheme{default}
\setbeamercolor{background canvas}{bg=black}

\setbeamercolor{normal text}{fg=white}
\setbeamercolor{block body}{bg=black!75,fg=white}

\setbeamercolor{frametitle}{bg=blue!80,fg=white}
\setbeamercolor{framesubtitle}{bg=blue!50,fg=white}

\setbeamercovered{dynamic}
  % or whatever (possibly just delete it)

\setbeamerfont{footnote}{size=\tiny}
}


\usepackage[english]{babel}

% or whatever

\usepackage[utf8]{inputenc}
\usepackage[T1]{fontenc}
% Or whatever. Note that the encoding and the font should match. If T1
% does not look nice, try deleting the line with the fontenc.


\title[Short Paper Title] % (optional, use only with long paper titles)
{Complex dynamical behaviors in a discrete
eco-epidemiological model with disease in
prey}

\subtitle
{Zengyun Hu, Zhidong Teng, Chaojun Jia, Long Zhang and Xi Chen}

% - Use the \inst command only if there are several affiliations.
% - Keep it simple, no one is interested in your street address.

\date[2026] % (optional, should be abbreviation of conference name)
{Presentazione esame "Sistemi complessi"}
% - Either use conference name or its abbreviation.
% - Not really informative to the audience, more for people (including
%   yourself) who are reading the slides online


\institute[Universities of Somewhere and Elsewhere] % (optional, but mostly needed)
{
Università degli studi di Trieste
}

% \subject{Theoretical Computer Science}
% This is only inserted into the PDF information catalog. Can be left
% out. 

\author % (optional, use only with lots of authors)
{Samuele Serri}

% If you have a file called "university-logo-filename.xxx", where xxx
% is a graphic format that can be processed by latex or pdflatex,
% resp., then you can add a logo as follows:

% \pgfdeclareimage[height=0.5cm]{university-logo}{university-logo-filename}
% \logo{\pgfuseimage{university-logo}}



% Delete this, if you do not want the table of contents to pop up at
% the beginning of each subsection:
% \AtBeginSubsection[]
% {
%   \begin{frame}<beamer>{Outline}
%     \tableofcontents[currentsection,currentsubsection]
%   \end{frame}
% }


% If you wish to uncover everything in a step-wise fashion, uncomment
% the following command: 

%\beamerdefaultoverlayspecification{<+->}


\begin{document}

\begin{frame}
  \titlepage
\end{frame}

\begin{frame}{Outline}
  \tableofcontents
  % You might wish to add the option [pausesections]
\end{frame}


% Structuring a talk is a difficult task and the following structure
% may not be suitable. Here are some rules that apply for this
% solution: 

% - Exactly two or three sections (other than the summary).
% - At *most* three subsections per section.
% - Talk about 30s to 2min per frame. So there should be between about
%   15 and 30 frames, all told.

% - A conference audience is likely to know very little of what you
%   are going to talk about. So *simplify*!
% - In a 20min talk, getting the main ideas across is hard
%   enough. Leave out details, even if it means being less precise than
%   you think necessary.
% - If you omit details that are vital to the proof/implementation,
%   just say so once. Everybody will be happy with that.

\section{Introduction}

\subsection{Eco-epidemiological models}

\begin{frame}{Eco-epidemiological models.}
  % - A title should summarize the slide in an understandable fashion
  %   for anyone how does not follow everything on the slide itself.
\begin{itemize}
  \item Combine ecological models of population dynamics with epidemiological models.
  \item 
\end{itemize}
\end{frame}

\subsection{The model studied}

\begin{frame}{Discrete time eco-epidemiological model with disease in prey}
  \begin{block}{Model Equations}
  \begin{align}
    S(t + 1) &= S(t)\exp\left\{r\left(1 - \frac{S(t) + I(t)}{K}\right) - \beta I(t)\right\} \\
    I(t + 1) &= I(t)\exp\left\{\beta S(t) - c - \frac{bY(t)}{mY(t) + I(t)}\right\} \\
    Y(t + 1) &= Y(t)\exp\left\{\frac{kbI(t)}{mY(t) + I(t)} - d\right\}
  \end{align}
  \end{block}
  Where $S(t)$, $I(t)$ and $Y(t)$ denote the susceptible prey, infected prey and predator populations at time $t$, respectively.
  % You can create overlays\dots
  % \begin{itemize}
  % \item using the \texttt{pause} command:
  %   \begin{itemize}
  %   \item
  %     First item.
  %     \pause
  %   \item    
  %     Second item.
  %   \end{itemize}
  % \item
  %   using overlay specifications:
  %   \begin{itemize}
  %   \item<3->
  %     First item.
  %   \item<4->
  %     Second item.
  %   \end{itemize}
  % \item
  %   using the general \texttt{uncover} command:
  %   \begin{itemize}
  %     \uncover<5->{\item
  %       First item.}
  %     \uncover<6->{\item
  %       Second item.}
  %   \end{itemize}
  % \end{itemize}
\end{frame}

\begin{frame}{Parameters}
  \begin{itemize}
    \item $r$: intrinsic birth rate of the prey population.
    \item $K$: carrying capacity for the prey population.
    \item $\beta$: transmission coefficient of the disease.
    \item $c$: death rate of the infected prey.
    \item $b$: predation coefficient.
    \item $m$: ratio-dependent rate. \footnote{{See: Xiao, Y, Chen, L: A ratio-dependent predator-prey model with disease in the prey. Appl. Math. Comput. 131, 397-414 (2002)}}
    \item $k$: coefficient in conversing prey into predator offspring.
    \item $d$: death rate of the predator.
  \end{itemize}
  $r, k, b, \beta, K, m, c, d > 0$.
\end{frame}

\section{Equilibria}

\subsection{Existence}

\begin{frame}{Basic Reproduction Number.}
  We can compute the \textit{basic reproduction number}, that is the expected number of cases directly generated by one case in a population where all individuals are susceptible to infection.
  \vspace*{0.8 cm}
  \begin{block}{$\mathcal{R}_0$: Basic Reproduction Number}
    \begin{equation}
      \mathcal{R}_0 = \frac{K\beta}{c}
    \end{equation}
  \end{block}
\end{frame}

\begin{frame}{The number of equilibria depends on $\mathcal{R}_0$.}
  \begin{columns}[T]
    \column{0.5\textwidth}
    \begin{block}{$\mathcal{R}_0 < 1$}
      \begin{equation}
        \mathcal{R}_0 = \frac{K\beta}{c} < 1
      \end{equation}
      One equilibrium:
      \begin{itemize}  
        \item $E_1 = (K,0,0)$
      \end{itemize}
    \end{block}
    
    \column{0.5\textwidth}
    \begin{block}{$\mathcal{R}_0 > 1$}
      \begin{equation}
        \mathcal{R}_0 = \frac{K\beta}{c} > 1
      \end{equation}
      Always at least two equilibria:
      \begin{itemize}
        \item $E_1 = (K,0,0)$
        \item $E_2 = \left(\frac{c}{\beta}, \frac{rK}{r + K\beta}(1 - \frac{c}{K\beta}), 0\right)$
      \end{itemize}
    \end{block}
  \end{columns}
\end{frame}

\begin{frame}{Existence of the positive endemic equilibrium.}
  If $kb > d$ and $mk(K\beta - c) - (kb - d) > 0$, then there exists a positive endemic equilibrium;
  \vspace*{0.8 cm}
  \begin{block}
  {$E_3 = (S^*, I^*, Y^*)$}
  where:
  \begin{itemize}
    \item $S^* = \frac{cmk + kb - d}{mk\beta}$,
    \item  $I^* = \frac{r}{r + K\beta}(K - S^*)$,
    \item  $Y^* = \frac{kb - d}{md}I^*$
  \end{itemize}
\end{block}
\end{frame}



\subsection{Stability}

\begin{frame}{Stability of $E_1$.}
  \begin{block}{Theorem 1}
    \begin{enumerate}
      \item if $\mathcal{R}_0 < 1$ and $0 < r < 2$, $E_1$ is locally asymptotically stable; 
      \item if $\mathcal{R}_0 < 1$ and $r > 2$, $E_1$ is unstable;
      \item if $\mathcal{R}_0 > 1$, $E_1$ is unstable.
    \end{enumerate}
  \end{block}
\end{frame}

\begin{frame}{Stability of $E_2$.}
  \begin{block}{Theorem 2}
    Let $\mathcal{R}_0 > 1$, then $E_2$ is locally asymptotically stable if the following conditions hold:
    \begin{align*}
      &bk < d,
      &(r - 4)c < 4, \hspace*{1cm} 
      &\frac{cr(c + 2)}{4 + cr} < K\beta < 1 + c
    \end{align*}
  \end{block}
These results are obtained by studying the eigenvalues of the Jacobian matrix evaluated at the equilibria.
\begin{equation*}
  J(E_2) = \begin{pmatrix}
    1 - \frac{rS}{K} & -S(\beta + \frac{r}{K}) & 0 \\
    \beta I & 1 & -b \\
    0 & 0 & e^{bk - d}
  \end{pmatrix}
\end{equation*}
\end{frame}

\begin{frame}{Stability of $E_3$.}
  \begin{block}{Theorem 3}
    Let $\mathcal{R}_0 > 1, kb > d$ and $mk(K\beta - c) - (kb - d) > 0$, then $E_3$ is locally asymptotically stable if one of the following conditions hold:
    \begin{enumerate}
      \item $\Delta \leq 0$, $P(- 1) < 0$ and $-1 < \lambda_{1,2} < 1$;
      \item $\Delta > 0$, $P(-1) < 0$ and $-1 < \lambda_{2,3} < 1$.
    \end{enumerate}
  \end{block}
\begin{equation*}
  J(E_3) = \begin{pmatrix}
    1 - \frac{r}{K}S^* & -S^*(\beta + \frac{r}{K}) & 0 \\
    \beta I^* & 1 + \frac{bI^*Y^*}{(mY^* + I^*)^2} &  -\frac{b(I^*)^2}{(mY^* + I^*)^2}\\
    0 & \frac{bkm(Y^*)^2}{(mY^* + I^*)^2} & 1 - \frac{bkmI^*Y^*}{(mY^* + I^*)^2} \\
  \end{pmatrix}
\end{equation*}
And {\small$P(\lambda) = \lambda^3 + a_1\lambda^2 + a_2\lambda + a_3$} \\
{\tiny $a_1 = -(J_{11} + J_{22} + J_{33}), \newline 
a_2 = J_{11}(J_{22} + J_{33}) + J_{22}J_{33} - J_{23}J_{32} - J_{12}J_{21}, \newline
a_3 = det(J(E_3))$.}
\end{frame}

%%%%  PART II

\section{Numerical simulations}
\subsection{E1, Equilibrium without predator and disease-free prey}

\begin{frame}{Period Doubling Bifurcation for $r > 2$.}
\begin{figure}
  \centering
  \includegraphics[width=.6\textwidth]{Imgs/EX1/Ex1_bifurcation.png}
  \caption{$b = 0.2, c = 0.6, d = 0.12, k  = 0.1, m = 0.2, \beta = 0.05, K = 8, r \in [0.01,4], S0 = 4, I0 = 0.5, Y0 = 0.1$}
\end{figure}
\end{frame}

\begin{frame}
  \begin{equation*}
      \mathcal{R}_0 = \frac{K\beta}{c} = \frac{8 \times 0.05}{0.6} \approx 0.67 < 1
  \end{equation*}
  \begin{figure}
    \centering
    \begin{minipage}{0.45\textwidth}
      \centering
      \includegraphics[width=\textwidth]{Imgs/EX1/Ex1_phase_diagram_r_equal_to_1.png}
      \caption{Phase space diagram for $r=1$.}
    \end{minipage}
    \hfill
    \begin{minipage}{0.45\textwidth}
      \centering
      \includegraphics[width=\textwidth]{Imgs/EX1/Ex1_phase_diagram_r_equal_to_2.png}
      \caption{Phase space diagram for $r=2$.}
    \end{minipage}
  \end{figure}
\end{frame}

\subsection{E2, equilibrium with disease, but no predator}

\begin{frame}{Flip bifurcation and chaos}
  \begin{figure}
    \centering
    \includegraphics[width=.6\textwidth]{Imgs/EX2/Es2_bifurcation.png}
    \caption{$b = 0.15, c = 0.1, d = 0.2, k  = 0.2, m = 0.3, \beta = 0.05, K = 4, r \in [0.001, 7], S0 = 2, I0 = 1.5, Y0 = 1$}
  \end{figure}
\end{frame}

\begin{frame}
  \begin{equation*}
      \mathcal{R}_0 = \frac{K\beta}{c} = \frac{4 \times 0.05}{0.1} = 2 > 1
  \end{equation*}
  \begin{flushright}
    \begin{figure}
      \centering
      \begin{minipage}{0.45\textwidth}
        \centering
        \includegraphics[width=\textwidth]{Imgs/EX2/Phase_space_r_equal_002.png}
        \caption{Phase space diagram for $r=0.02$.}
      \end{minipage}
      \hfill
      \begin{minipage}{0.45\textwidth}
        \centering
        \includegraphics[width=\textwidth]{Imgs/EX2/Phase_space_r_equal_6_8.png}
        \caption{Phase space diagram for $r=6.05$.}
      \end{minipage}
    \end{figure}
    A 3-period orbit appears for $r \approx 6.05$.
  \end{flushright}
\end{frame}

\begin{frame}{\small{$0 < c \leq 1$ Hopf bifurcation and chaos. $1 < c < 1.633$ $S$ increases, $I$ decreases. $c \geq 1.633$ flip bifurcation and chaos.}}
  \begin{figure}
    \centering
    \includegraphics[width=.6\textwidth]{Imgs/EX3/Ex3_bifurcation.png}
    \caption{\tiny{$b = 0.2, d = 0.2, k = 0.2, m = 0.5, r = 3, \beta = 0.2, K = 10, c \in [0.5, 2.5], S0 = 6, I0 = 1, Y0 = 1$}}
  \end{figure}
\end{frame}

\begin{frame}
  \begin{equation*}
      \mathcal{R}_0 = \frac{K\beta}{c} = \frac{10 \times 0.2}{c} = \frac{2}{c}
  \end{equation*}
  \begin{flushright}
    \begin{figure}
      \centering
      \begin{minipage}{0.45\textwidth}
        \centering
        \includegraphics[width=\textwidth]{Imgs/EX3/Ex3_phase_space_c_equal_1_5.png}
        \caption{Phase space diagram for $c=1.5$.}
      \end{minipage}
      \hfill
      \begin{minipage}{0.45\textwidth}
        \centering
        \includegraphics[width=\textwidth]{Imgs/EX3/Ex3_phase_space_c_equal_1.png}
        \caption{Phase space diagram for $c=1$.}
      \end{minipage}
    \end{figure}
  \end{flushright}
\end{frame}



\begin{frame}{\small{$E_2$ stable for $2 < K < 7$, flip bifurcation and Hopf bifurcation for $K \geq 7$.}}
  \begin{figure}
    \centering
    \includegraphics[width=.6\textwidth]{Imgs/EX4/EX4_bifurcation.png}
    \caption{$b = 0.1, c = 0.4, d = 0.2, k  = 0.2, m = 0.5, r = 0.2, \beta = 0.2, K \in [2,8], S0 = 1, I0 = 0.5, Y0 = 0.2$}
  \end{figure}
\end{frame}

\begin{frame}
  \begin{equation*}
      \mathcal{R}_0 = \frac{K\beta}{c} = \frac{K \times 0.2}{0.4} = \frac{K}{2}
  \end{equation*}
  \begin{flushright}
    \begin{figure}
      \centering
      \begin{minipage}{0.45\textwidth}
        \centering
        \includegraphics[width=\textwidth]{Imgs/EX4/Ex4_phase_space_K_equal_4.png}
        \caption{Phase space diagram for $K=4$.}
      \end{minipage}
      \hfill
      \begin{minipage}{0.45\textwidth}
        \centering
        \includegraphics[width=\textwidth]{Imgs/EX4/Ex4_phase_space_K_equal_8.png}
        \caption{Phase space diagram for $K=8$.}
      \end{minipage}
    \end{figure}
  \end{flushright}
\end{frame}


\subsection{E3, endemic equilibrium}


\begin{frame}{\small{$0.15 < b < 0.28$ Hopf bifurcation. $ b \geq 0.28 < 0.57$  stable equilibrium.}}
  \begin{figure}
    \centering
    \includegraphics[width=.6\textwidth]{Imgs/EX5/Ex5_bifurcation.png}
    \caption{$c = 0.1, d = 0.02, k  = 0.3, m = 0.4, r = 1.2, \beta = 0.25, K = 6, b \in [0.15,7], S0 = 2, I0 = 1.5, Y0 = 1$}
  \end{figure}
\end{frame}

\begin{frame}
  \begin{equation*}
      \mathcal{R}_0 = \frac{K\beta}{c} = \frac{6 \times 0.25}{0.1} = 15
  \end{equation*}
  \begin{flushright}
    \begin{figure}
      \centering
      \begin{minipage}{0.45\textwidth}
        \centering
        \includegraphics[width=\textwidth]{Imgs/EX5/ex5_phase_diagram_b_equal_0_6.png}
        \caption{Phase space diagram for $b=0.6$.}
      \end{minipage}
      \hfill
      \begin{minipage}{0.45\textwidth}
        \centering
        \includegraphics[width=\textwidth]{Imgs/EX5/ex5_phase_diagram_b_equal_0_2.png}
        \caption{Phase space diagram for $b=0.2$.}
      \end{minipage}
    \end{figure}
  \end{flushright}
\end{frame}


\begin{frame}{\small{$0.15 < k < 0.34$ Hopf bifurcation. $ 1 > k \geq 0.34$ $S,I$ increasing and $Y$ decreasing.}}
  \begin{figure}
    \centering
    \includegraphics[width=.6\textwidth]{Imgs/EX6/Ex6_bifurcation.png}
    \caption{$b = 0.3, c = 0.1, d = 0.04, m = 0.3, r = 1.2, \beta = 0.25, K = 6, k \in [0.15, 1], S0 = 2, I0 = 1.5, Y0 = 1$}
  \end{figure}
\end{frame}

\begin{frame}
  \begin{equation*}
      \mathcal{R}_0 = \frac{K\beta}{c} = \frac{6 \times 0.25}{0.1} = 15
  \end{equation*}
  \begin{flushright}
    \begin{figure}
      \centering
      \begin{minipage}{0.45\textwidth}
        \centering
        \includegraphics[width=\textwidth]{Imgs/EX6/ex_6_pahse_diagram_k_equal_0_3.png}
        \caption{Phase space diagram for $k=0.3$.}
      \end{minipage}
      \hfill
      \begin{minipage}{0.45\textwidth}
        \centering
        \includegraphics[width=\textwidth]{Imgs/EX6/ex_6_pahse_diagram_k_equal_0_8.png}
        \caption{Phase space diagram for $k=0.8$.}
      \end{minipage}
    \end{figure}
  \end{flushright}
\end{frame}


\section*{Summary}

\begin{frame}{Summary}

  % Keep the summary *very short*.
  \begin{itemize}
  \item
    The \alert{first main message} of your talk in one or two lines.
  \item
    The \alert{second main message} of your talk in one or two lines.
  \item
    Perhaps a \alert{third message}, but not more than that.
  \end{itemize}
  
  % The following outlook is optional.
  \vskip0pt plus.5fill
  \begin{itemize}
  \item
    Outlook
    \begin{itemize}
    \item
      Something you haven't solved.
    \item
      Something else you haven't solved.
    \end{itemize}
  \end{itemize}
\end{frame}



% All of the following is optional and typically not needed. 
\appendix
\section<presentation>*{\appendixname}
\subsection<presentation>*{For Further Reading}

\begin{frame}[allowframebreaks]
  \frametitle<presentation>{For Further Reading}
    
  \begin{thebibliography}{10}
    
  \beamertemplatebookbibitems
  % Start with overview books.

  \bibitem{Author1990}
    A.~Author.
    \newblock {\em Handbook of Everything}.
    \newblock Some Press, 1990.
 
    
  \beamertemplatearticlebibitems
  % Followed by interesting articles. Keep the list short. 

  \bibitem{Someone2000}
    S.~Someone.
    \newblock On this and that.
    \newblock {\em Journal of This and That}, 2(1):50--100,
    2000.
  \end{thebibliography}
\end{frame}

\end{document}